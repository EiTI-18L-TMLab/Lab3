\documentclass[a4paper,titlepage,11pt,floatssmall]{mwrep}
\usepackage[left=2.5cm,right=2.5cm,top=2.5cm,bottom=2.5cm]{geometry}
\usepackage[OT1]{fontenc}
\usepackage{polski}
\usepackage[utf8]{inputenc}
\usepackage{amsmath}
\usepackage{amssymb}
\usepackage{graphicx}
\usepackage{mathrsfs}
\usepackage{rotating}
\usepackage{pgfplots}
\usetikzlibrary{pgfplots.groupplots}

\usepackage{siunitx}

\usepackage{float}
\definecolor{szary}{rgb}{0.95,0.95,0.95}
\sisetup{detect-weight,exponent-product=\cdot,output-decimal-marker={,},per-mode=symbol,binary-units=true,range-phrase={-},range-units=single}

\SendSettingsToPgf
\title{\bf Sprawozdanie z laboratorium nr 3\\ Projekt licznika 8-bitowego przy użyciu mikrokontrolera z rodziny MSP430F16x  \vskip 0.1cm}
\author{Jakub Sikora \and Konrad Winnicki \and Marcin Dolicher}
\date{\today}
\pgfplotsset{compat=1.15}	
\begin{document}


\makeatletter
\renewcommand{\maketitle}{\begin{titlepage}
		\begin{center}{\LARGE {\bf
					Wydział Elektroniki i Technik Informacyjnych}}\\
			\vspace{0.4cm}
			{\LARGE {\bf Politechnika Warszawska}}\\
			\vspace{0.3cm}
		\end{center}
		\vspace{5cm}
		\begin{center}
			{\bf \LARGE Technika Mikroprocesorowa \vskip 0.1cm}
		\end{center}
		\vspace{1cm}
		\begin{center}
			{\bf \LARGE \@title}
		\end{center}
		\vspace{2cm}
		\begin{center}
			{\bf \Large \@author \par}
		\end{center}
		\vspace*{\stretch{6}}
		\begin{center}
			\bf{\large{Warszawa, \@date\vskip 0.1cm}}
		\end{center}
	\end{titlepage}
	}
\makeatother
\maketitle

\tableofcontents


\chapter{Zadanie laboratoryjne}
\section{Polecenie}
\indent{} Celem trzeciego zadania laboratoryjnego było zaprojektowanie, złożenie, zaprogramowanie i przetestowanie układu z mikrokontrolerem MSP430F16x tak aby działał on jako licznik 8-bitowy zliczający w dół liczbę wciśnięć programowo odszumionego przycisku monostabilnego, który umożliwiał asynchroniczne ładowanie liczby do pamięci z blokowaniem zliczania. Ładowanie liczby powinno odbyć się przy pomocy przerwania maskowalnego. Liczby powinny być przechowywane w Naturalnym Kodzie Binarnym, w skrócie NKB.

\section{Szczegółowe uwagi}
\indent W odróżnieniu do zadania z drugiego laboratorium, wyświetlanie zawartości licznika powinno odbyć się na dwóch wyświetlaczach 7-segmentowych. Liczby powinny być wyświetlane w postaci szesnastkowej. Ładowanie zawartości licznika ma odbywać się za pomocą nastawników hex, a sterowanie licznikiem ma odbywać się za pomocą dwóch przycisków, z czego co najmniej jeden ma być obsługiwany w przerwaniu. Ostatecznie, w momencie gdy procesor nic nie robi, powinien przejść w tryb oszczędzania energii LPM0.

\chapter{Projekt licznika}

\section{Opis sprzętu}
\indent{} Zmiana platformy z procesora Z80 na mikrokontroler MSP43016x znacząco ułatwiało zadanie i pozwoliło na dokonanie kilku ulepszeń w projekcie. Mikrokontroler ma wbudowane układy podtrzymujące w swoich portach wejścia/wyjścia dlatego też nie są potrzebne dodatkowe układy złożone z buforów trójstanowych. Schemat ideowy układu znajduje się na rysunku poniżej.

\begin{figure}[th]
\centering
\includegraphics[width=\textwidth]{ideowy}
\caption{Schemat ideowy}
\end{figure}

Do portu pierwszego podłączyliśmy zestaw ośmiu przycisków monostabilnych za pomocą których użytkownik może obsługiwać licznik. Do portu trzeciego podłączyliśmy obrotowe nastawniki liczby w kodzie heksadecymalnym. Ostatecznie do portu czwartego podłączyliśmy układ z dwoma wyświetlaczami siedmiosegmentowymi wraz z odpowiednim dekoderem. W porównaniu z rozwiązaniem opartym na procesor Z80, zastosowanie mikrokontrolera znacząco upraszcza ostateczny układ. Mniejsza ilość peryferiali jest możliwa dzięki wewnętrznym rejestrom, które są przypisane do odpowiednich portów.

\section{Opis oprogramowania}
\indent W trakcie projektowania oprogramowania skupiliśmy się na tym aby nasze rozwiązanie pobierało jak najmniej energii. W trakcie oczekiwania na wciśnięcie przycisku, procesor miał pozostawać w stanie ograniczonego poboru mocy, podczas którego zegary procesora pozostawały wyłączone. Wyjście z tego stanu odbywało się za pomocą przerwania maskowalnego. Po przyjęciu przerwania i jego obsłużeniu, program miał zadecydować czy ma coś jeszcze do zrobienia czy jednak może wrócić do stanu niskiego poboru mocy. W przypadku asynchronicznego ładowania, po wczytaniu liczby procesor mógł spokojnie przejść w stan oczekiwania. Po wciśnięciu przycisku \texttt{CLK}, mikrokontroler powinien rozpocząć procedurę odszumiania przycisku.

\subsection{Odszumianie przycisku}
\indent W momencie wciśnięcia przycisku \texttt{CLK} program przechodzi do specjalnej procedury podczas której zliczane są kolejne niskie stany przycisku. Po pojawieniu się kolejnych 256 niskich stanów pod rząd, program uznaje że przycisk się już ustabilizował i traktuje to jako jedno wciśnięcie. Program następnie oczekuje na kolejne 256 wysokich stanów, które świadczą o puszczeniu i ustabilizowaniu przycisku. Po odnotowaniu takiej sytuacji, procesor przechodzi w tryb zmniejszonego poboru mocy i oczekuje na kolejne przerwanie pochodzące albo od przycisku \texttt{LD} albo \texttt{CLK}

\subsection{Przerwania}
\indent Procedura przerwania zostaje wywołana przy wciśnięciu przycisku \texttt{LD} lub \texttt{CLK}. W trakcie tej procedury, sprawdzane jest od którego przycisku przyszło przerwanie. Podczas obsługi przycisku ładowania, pobierana jest wartość z nastawników hex a następnie zostaje ona załadowana do pamięci i wypuszczona na wyświetlacze. Ostatecznie po odświeżeniu, procesor wraca do trybu zmniejszonego poboru mocy. W przypadku przyjęcia przerwania od przycisku zegarowego, program przechodzi do głównej pętli programu a do trybu zmniejszonego poboru mocy przechodzi tylko w po zakończeniu obsługi wciśnięcia przycisku.

\subsection{Kod programu}
\medskip
\noindent
;-------------------------------------------------------------------------------\\
; MSP430 Assembler Code Template for use with TI Code Composer Studio\\
;\\
;\\
;-------------------------------------------------------------------------------\\
            .cdecls C,LIST,"msp430.h"       ; Include device header file\\
            \\
;-------------------------------------------------------------------------------\\
            .def    RESET                   ; Export program entry-point to\\
                                            ; make it known to linker.\\
;-------------------------------------------------------------------------------\\
            .text                           ; Assemble into program memory.\\
            .retain                         ; Override ELF conditional linking\\
                                            ; and retain current section.\\
            .retainrefs                     ; And retain any sections that have\\
                                            ; references to current section.\\
\\
;-------------------------------------------------------------------------------\\
\\
\\
\\
;-------------------------------------------------------------------------------\\
; Main loop here\\
;-------------------------------------------------------------------------------\\
RESET:\\
        MOV.W   \#\_{}\_{}STACK\_END,SP        ; set up stack\\
MAIN:\\
	MOV.B \#0FFh, \&P2DIR\\
\\
	MOV.W   \#WDTPW|WDTHOLD,\&WDTCTL  ;Stop watchdog timer\\
    MOV.B   \#0FFh, \&P4DIR\\
    MOV.B   \#0FFh, \&P4OUT\\
    MOV.B   \#000h, \&P1IFG\\
    MOV.B   \#060h, \&P1IE    ; aktywacja przerwan od CLK i LD\\
    MOV.B	\#060h, \&P1IES\\
\\
	MOV.B \&P3IN, R4 		; inicjacja wartosci licznika\\
	MOV.B R4, \&P4OUT		; odswiezenie licznika\\
	MOV \#01Fh, R6 	        ; ilosc cykli debounce\\
	MOV R6, R5 	        	; ustawienie licznika debounce\\
	BIC \#0FFh, R7 	        ; wybrana obsluga zbocza opadajacego, main sygnalizuje ukonczenie zadan\\
\\
    BIS \#GIE+CPUOFF,SR 		; aktywacja LPM1 - program zatrzymuje sie w tym miejscu\\
MAIN\_ LOOP: 		        	; poczatek głównej petli\\
    BIC \#02h, R7\\
\\
	; poczatek obslugi licznika\\
    BIT \#01h, R7\\
	JNZ LOOP\_{}POSITIVE       ; wybranie obslugi konkretnego zbocza\\
LOOP\_{}NEGATIVE:	            ; obsluga zbocza opadajacego\\
	BIT.B \#040h, \&P1IN		; stan przycisku CLK\\
	JZ IF\_{}1\\
	MOV R6, R5 	        	; jesli przycisk niewcisniety, zaladuj licznik debounce\\
    BIS \#02h, R7            ; sygnalizacja pracy w toku\\
    BIC.B \#040h, \&P1IE      ; wylaczenie przerwania od CLK(P1.6)\\
    BIS.B \#040h, \&P1IES     ; wybranie zbocza opadajacego\\
    BIC.B \#040h, \&P1IFG     ; wyzerowanie zadania przerwania od CLK\\
    BIS.B \#040h, \&P1IE      ; wlaczenie przerwania od CLK\\
	JMP COUNTER\_{}END\\
IF\_{}1:\\
	JNZ IF\_{}2\\
	DEC R5 		        	; jesli przycisk wcisniety, dekrementuj licznik debounce\\
IF\_{}2:\\
	CMP \#0, R5 				; czy licznik debounce równy zero\\
	JZ IF\_{}2\_{}CONTINUE        ; jesli licznik debounce zerowy, dalsza obsluga\\
    BIS \#02h, R7            ; sygnalizacja pracy w toku\\
	JMP COUNTER\_{}END\\
IF\_{}2\_{}CONTINUE:\\
    BIS \#02h, R7            ; sygnalizacja pracy w toku\\
	DEC R4		        	; wykryto poprawne zbocze opadajace, akcja licznika\\
	MOV.B R4, \&P4OUT		; odswiezenie LED\\
	BIS \#01h, R7            ; wybranie obslugi zbocza narastajacego\\
	JMP COUNTER\_{}END         ; skok na koniec obslugi licznika\\
LOOP\_{}POSITIVE:	            ; obsluga zbocza narastajacego\\
    BIS \#02h, R7            ; sygnalizacja pracy w toku\\
	BIT.B \#040h, \&P1IN		; stan przycisku CLK\\
	JNZ IF\_{}3\\
	MOV R6, R5 	        	; jesli przycisk wcisniety, zaladuj licznik debounce\\
IF\_{}3:\\
	JZ IF\_{}4\\
	DEC R5		        	; jesli przycisk niewcisniety, dekrementuj licznik debounce\\
IF\_{}4:\\
	CMP \#0, R5	        	; odswiezenie stanu flag\\
	JZ IF\_{}4\_{}CONTINUE        ; jesli licznik debounce zerowy, dalsza obsluga W IF\_{}4\_{}CONTINUE\\
    BIS \#02h, R7            ; sygnalizacja pracy w toku\\
	JMP COUNTER\_{}END\\
IF\_{}4\_{}CONTINUE:\\
	BIC \#01h, R7            ; wybranie obslugi zbocza opadajacego\\
    BIC.B \#040h, \&P1IE      ; wylaczenie przerwania od CLK(P1.6)\\
    BIS.B \#040h, \&P1IES     ; wybranie zbocza opadajacego\\
    BIC.B \#040h, \&P1IFG     ; wyzerowanie zadania przerwania od CLK\\
    BIS.B \#040h, \&P1IE      ; wlaczenie przerwania od CLK\\
    MOV R6, R5				; zaladowanie licznika debounce\\
COUNTER\_{}END:				; koniec obslugi licznika\\
\\
    ; dalsze operacje...\\
\\
    ; jesli zadna operacja nie zglosila pracy w toku przejdz w tryb uspienia\\
    BIT \#02h, R7\\
    JNZ MAIN\_{}LOOP\\
	BIS \#GIE+CPUOFF,SR ; aktywacja LPM3 - program zatrzymuje sie w tym miejscu\\
\\
    JMP MAIN\_{}LOOP ; powrót do poczatku glównej petli    \\
\\
P1\_{}INT\_{}VECTOR:                          ; procedura obslugi przerwania P1\\
        BIT.B \#020h, \& P1IFG             ; sprawdzenie czy aktywne jest zadanie przerwania lub niski stan na LD\\
        JNZ P1\_{}INT\_{}VEC\_{}LOAD\\
        BIT.B \#020h, \& P1IN      ;\\
        JNZ P1\_{}INT\_{}VEC\_{}LOAD\_{}DONE\\
P1\_{}INT\_{}VEC\_{}LOAD:                        ; ciagle ladowanie gdy niski stan LD\\
        BIC.B \#020h, P1IFG              ; zgaszenie flagi przerwania od LD(P1.5)\\
		MOV.B \&P3IN, R4 	        ; zaladowanie nowej wartosci licznika\\
		MOV.B R4, \&P4OUT	        ; odswiezenie wyswietlania\\
        JMP P1\_{}INT\_{}VECTOR\\
P1\_{}INT\_{}VEC\_{}LOAD\_{}DONE:\\
        BIT.B \#040h, \&P1IFG             ; sprawdzenie przerwania od CLK(P1.6)\\
        JZ P1\_{}INT\_{}VEC\_{}CLK\_{}DONE\\
        BIC.B \#040h, \&P1IE              ; interesuje nas tylko pierwsze przerwanie od CLK\\
        BIC.B \#040h, \&P1IFG             ; zgaszenie flagi przerwania od CLK\\
        BIC \#CPUOFF+SCG1+SCG0,0(SP)     ; powrót z przerwania ze zmienionym SR\\
P1\_{}INT\_{}VEC\_{}CLK\_{}DONE:\\
        RETI\\
\\
\\
;-------------------------------------------------------------------------------\\
; Stack Pointer definition\\
;-------------------------------------------------------------------------------\\
            .global \_{}\_{}STACK\_{}END\\
            .sect   .stack\\
\\
;-------------------------------------------------------------------------------\\
; Interrupt Vectors\\
;-------------------------------------------------------------------------------\\
            .sect   ".reset"                ; MSP430 RESET Vector\\
            .short  RESET\\
\\
            .sect ".int04"\\
            .short P1\_{}INT\_{}VECTOR 
            
       
\end{document}